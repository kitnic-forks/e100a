The PS-1 is a bench power supply intended for powering general analog and
digital devices under test (DUTs). It is designed entirely using
discrete semiconductors, and is designed to be as usable as a traditional power
supply based on precision operational amplifiers.

\subsubsection{Assembly}
To assemble the PS-1, you must handle and solder modern, surface-mount
devices (SMD). This manual will not teach you how to do this. A good
resource for learning how to solder SMD parts is the excellent soldering
video tutorial series by Dave Jones of EEVblog, which can be found on
YouTube:

\begin{center}
\begin{tabular}{lc}
\#180 --- Soldering Tutorial Part 1 --- Tools &
    \url{http://www.youtube.com/watch?v=J5Sb21qbpEQ} \\
\#183 --- Soldering Tutorial Part 2 &
    \url{http://www.youtube.com/watch?v=fYz5nIHH0iY} \\
\#186 --- Soldering Tutorial Part 3 --- Surface Mount &
    \url{http://www.youtube.com/watch?v=b9FC9fAlfQE} \\
\end{tabular}
{\small\it Please support Dave by disabling any ad blockers while watching
    these videos. This is how he makes his living.}
\hrule
\end{center}

You also may need to source the components to build the kit, as not all
incarnations provide all of the parts. A separate packet detailing parts which
were included and parts which were not should be attached; this will include
bills of materials (BOMs) directly compatible with multiple major parts
distributors to ease the process.

\subsubsection{Testing}
Testing the PS-1 requires only a multimeter of any kind and an oscilloscope
with at least 10 MHz bandwidth%
\footnote{Measuring some parameters requires more sophisticated equipment, but
I have done this already. If you build the PS-1 using the specified parts in
the specified manner, and you test it and find that it works, it should meet
these specifications.}.
You will build the supply one section at a
time, testing each individually, to isolate any problems to the exact section
which causes them. There is a troubleshooting guide in this booklet, but if you
encounter serious difficulty, you will find it necessary to read and understand
the Theory of Operation to fully debug the circuits.

\subsubsection{Voltage Warning!}
The ``low-voltage'' circuitry of this power supply contains voltages
ranging from 40~V to 50~V, which will probably not kill you, but could be
dangerous.  Do not touch the circuit while it is operational, and do not touch
anything powered by it until you have fully verified correct operation. You
should take care to fully determine and understand the paths by which dangerous
current can flow.

Assembling the PS-1 also requires you to make a small number of connections
that will operate at line voltage (120~V or 240~V AC, depending on your
country). It is \emph{your responsibility} to determine whether you can legally
perform this in your area, as I cannot be expected to know the electrical laws
of every jurisdiction. If this is prohibited or discouraged where you live,
I recommend finding someone licensed to do this wiring. Even if you \emph{can}
legally do this, I recommend politely asking someone with electrical experience
to check your work for safety.

\subsubsection{Energy Warning!}
Installing electrolytic capacitors backwards can make them \emph{pop!} If they
do this while your curious face is hovering over them, trying to figure out why
they are inflating like balloons, they can cause you physical injury. Any build
sections with capacitors in high-energy positions will advise you of this fact
before you power on the PS-1 for testing. I highly recommend covering the PS-1
and keeping your face away from it while connecting the power.

Likewise, resistors in any section of the circuit may burn if installed
incorrectly. This does not cause such a commotion as do the capacitors, but
it does smell very bad. You have been warned.
